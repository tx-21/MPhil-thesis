\chapter{Literature Review}
%\label{sec:ob_rel}
\section{Introduction to water quality control}
\subsection{Automated system for water quality control}
%%%%%%%%%%PLC and some real application cases in WTP
%Explain PLC better by \cite{WasteWaterTreatment2018}
%re organize this section
Programmable logic controller (PLC) is an industrial computer system designed for any process requiring a series of devices and equipment operates cohesively to achieve multiple purposes in manufacturing or treatment processes. The main components of PLC include a center process unit (CPU), input modules and output modules (I/O). CPU is responsible to process digital signals from input modules and send commands through output modules based on the control logics programmed on the PLC. For chemical dosing control in water treatment plants (WTPs), PLC system receives readings from turbidity and pH sensors and uses pumps to dose aluminum solution automatically \citep{andhareSCADAToolIncrease2014}. The PLC system with the capability of producing real-time output commands in response to the input sigals also makes it widely used in the wastewater treatment plants (WWTPs). For oxygen concentration control in the aeration tank, PLC system receives signals of dissovled oxygen (DO) detectors and transmits signals to open or close the electric butterfly valves to further alter the DO concentration \citep{zhuApplicationPLCSewage2017}. Although PLC systems are the most used system across industries for its easy programming and reliable control, PLC system is merely a device that can be programmed to control operative devices with on-off logic (i.e., a logic control with two states) and the capability of complex control is compromised. In reality, many WTPs or WWTPs have the need of precise control of the treatment processes. Being awared of the limitations of the PLC systems, a more advanced controller called proportional–interal–derivative (PID) controller for receiving analog signals was developed to obtain more sophisticated controls over the operative devices.

%%%%%%%%%%PID and examples
%PID control is used where greater levels of precision in control are required. It combines three control terms to give a single output to drive the setpoint. 
To react to rapidly-changing process conditions, a PID controller generates an output value based on continuous calculation of an error value e(t) as the difference between a desired setpoint (SP) and a measured process variable and applies a correction based on proportional, integral, and derivative terms. The use of the "P", "I", and "D" allows the system to quickly reach steady state with a feedback control system (i.e., the system output is returned to the system input which is included in the decision making process in PID controller). Generally speaking, a PID controller is a technology (i.e., a specialist algorithm) for contorlling a single device with more complex logics, while a PLC system is a physical system consists of different modules and capable of controlling dozens of devices only with two-state logic. In addition, A PID controller can be designed to operate on PLC device and provide a more precise control strategy to a designated device. In WWTPs, a single-variable feedback analog control loop in PID can be used to control the temperature in the activated sludge treatment by stablizing the system temperature in a shorter time \citep{badosDesignPIDControl2020}. The feedback control scheme is also applied in WTPs to adjust the addition of chlorine dosage (i.e., also known as the disinfection process, chlorination, or postchlorination) to reach the target concentration of free chlorine residual (FRC) \citep{wangCompositeControlPostChlorine2019}. Disinfection process is carried out in a chlorine contact tank which provides sufficient time for chlorine to disinfect pollutants. Since the chlorine added by the dosing device requires time to travel from the entry to the exit, the system output can only reflect the changes of water quality in a delayed time of 30 minutes (i.e., the designed time for water to travel in chlorine contact tank is usually 30 minutes or longer). In the case of chlorination, the lag of time makes feedback control difficult \citep{kobylinskiLineControlStrategies2006} as the system is delayed in responding to any sudden surge of the pollutants when it can only receive output at the end of the disinfection process. PID controllers in WWTPs also encounter similar challenges as the increasing complexity of water quality and stricter regulations on the discharged water quality. 

%%%%%%%%%%Transition to how the algorithms in PID can be replace by AI, ML, 
%Introduce to MPC, feedforward control in comparison with PID controller
%The use of mathematical modeling is not ideal, turning to the use of AI modeling
To tackle the difficulties encountered in process control system, many control strategies are proposed, such as feed forward-feedback control, linearized and optimal control, model-predictive control, and fuzzy control, etc \citep{demirFeedbackControlChlorine2014a}. Among the algorithms used in control strategies, Artificial Intelligence (AI) modeling has gained the most attentions in recent years compared to modeling based on mathematical models or empirical formulas. In WTPs or WWTPs, to fully understand the physical, biological, and chemical interactions in the treatment plants is very difficult. The unpredictable behaviors during the water treatment can be the significant changes of influent flow rate, flucutations of water quality, the complexity of biological treatment process, and the large time delay exists between this control variable and the process input, etc. Therefore, AI modeling shows a great potential in dealing with the highly complex conditions in the treatment process \citep{liRecentAdvancesArtificial2021}. In the next sections, the applications of different AI modeling methods will be discussed.

\subsection{Artificial Intelligence}
%AI include fuzzy logic, a subset of ML. ML includes SVM, a subset of DL. DL includes LSTM.
Artificial intelligence (AI) can perform cognitive tasks with the development of computational solutions. The concepts of AI are usually confused, in fact, AI is a very broad term and any kind of algorithms or models which involved in decision-making with computation fall in the domain of AI. For example, fuzzy logic and optimization algorithm are formulated with human design and computer decision making process. There are another subset of AI called machine learning (ML), but the process of generating a ML model is different to generating a fuzzy logic model. ML uses learning algorithms to generate a model via learning from historical or large amount of data without being explicitly programmed. ML algorithms can be classified into three categories, which are Supervised, Unsupervised, and Reinforcement learning. In the training process of supervised learning, input variable (x) and output variable(Y) we will provided, and model will learn from the provided dataset to map the x to the Y. A trained supervised model can generate a prediction for the response to the new data (i.e., also called the unseen data). Unsupervised learning is when the dataset is not labelled, the model can learn to infer patterns in the dataset without reference to the known outputs. This type of algorithm can find similarities and differences in the data. In reinforcement learning, models are designed to constantly interact with the environment in a try-and-error way and recieved rewards and punishments based on the purpose of the tasks. Generating a optimal action to achieve lowest penalties is the main function of a reinforcement learning model. In process control, supervised learning are frequently used in many senarios.

Regression is a supervised machine learning technique used to predict continuous values. A regression model can estimate the relationship between the input variables in the system and the output target from a given dataset, and then use the nonlinear relationship to map the unseen input data to a predicted output data. This type of application is sutiable for water quality prediction  \citep{librantzArtificialNeuralNetworks2018}, and sensor fault detection \citep{cecconiSoftSensingOnLine2021}, etc. 

Fuzzy logic (FL) control is still an effective strategy for process control, and this type of AI modeling is called reasoning. Fuzzy logic is described as an interpretative system in which objects or elements are related with borders not clearly defined, granting them a relative membership degree and not strict, as is customary in traditional logic. The typical architecture of a fuzzy controller, shown in Figure 3, consists of a fuzzifier, a fuzzy rule base, an inference engine, and a defuzzifier \citet{santinFuzzyControlModel2015} proposed a hybrid control system comprised of FL controller and model predictive control using optimizaion model to control the chlorine dosing in a WTP. FL controller and optimzation model fall in the domain of AI, which is excluded from the subset of ML.

Fuzzy logic (FL), a method based on multi-valued logic, uses fuzzy sets to study fuzzy judgement, which allows FL-based fuzzy inference systems to simulate the human brain to implement natural inference [40].The adaptive fuzzy neural inference system (ANFIS) composed of FL and ANN with an inference mechanism has high interpretability compared to common ANN. The combined model has been used to control coagulant dosing systems [41,42].

\subsection{Machine learning and deep learning}
In machine learning, popular models which are frequently used by the researchers for training predictive models are Supporting Vector Machine (SVM), Random Forest (RF), and Artificial Neural Networks (ANN). \citet{librantzArtificialNeuralNetworks2018} trained a RF model to predict the free residual chlorine concentration (FRC) in a WTP, and \citet{xuAlternativeLaboratoryTesting2021} built a RF-based model to predict total nitrogen concentration in water bodies. \citet{guoPredictionEffluentConcentration2015} compared the reliability and accuracy of an ANN model and a SVM model in predicting 1-day interval T-N concentration in a WWTP, and the results showed that RF model has higher accuracy while ANN model is more reliable for assisting decision-making process.

As the the computing power doubled every 18 months according to Moore's law. A subset of ML, Deep Learning (DL) becomes more accessible for sovling everyday issues. In simplicity, DL models can be defined as neural networks with more than two hidden layers (i.e., the model complexity increased and required more computing power to calculate). In DL, there are various types of architectures designed based on the type of problems. For image processing, Convolutional Neural Network (CNN) is designed to extract important features from the image vectors. Another popular DL architecture is Recurrent Neural Network (RNN), which is powerful in solving time series-related applications and Natural Language Processing (NLP) tasks \citep{liERNNDesignOptimization2018}. Although each architecture has their strengh in tackling different types of problems, both architectures can be used for a single task \citet{liPredictionFlowBased2022} built a regression CNN-RNN model for rainfall-runoff prediction. DL can be extremely powerful when multiple architectures ared fused into a single model to perform a specific task, which cannot be realized by machine leraning models. That being said, DL can achieve higher model performance in terms of the prediction accuracy compared to ML. 

\section{Water quality control with machine learning}
%%it's better to find papers that can provide comparison between DL and ML or traditional models
\subsection{Drinking water treatment plants}
%Background
A drinking water treatment plant (DWTPs) produces potable (i.e., drinking water) water for human consumptions by removing contaminants from the source water, such as lake or stream, or from an underground aquifer. The raw water enters DWTPs and goes through treatmet units of coagulation, flocculation, sedimentation, filtration, and disinfection in sequence as the primary treatment scheme in the conventional DWTPs \citep{liRecentAdvancesArtificial2021}. 

%detailed functions of the treatment process
1.8.1 Drinking Water Treatment Drinking water treatment plant could be classified into: – Disinfection plant which is used for high-quality water source to ensure that water does not contain pathogens – Filtration plant: this is usually used to treat surface water – Softening plant which is used to treat groundwater Typical filtration plant is shown in Fig. 5 which is designed to remove odors, color, and turbidity as well as bacteria and other contaminants. Filtration plant employs the following steps: a.Rapid mixing : where chemicals are added and rapidly dispersed through the water b.Flocculation : Chemicals like alum (aluminum sulfate) are added to the water both to neutralize the particles electrically and to make them come close to each other and form large particles called flocs that could more readily be settled out c.Sedimentation : During sedimentation, floc settles to the bottom of the water supply, due to its weight d.Filtration: Once the floc has settled to the bottom of the water supply, the clear water on top will pass through filters of varying compositions (sand, gravel, and charcoal) and pore sizes in order to remove fine particles that were not settled, such as dust, parasites, bacteria, viruses, and chemicals e.Disinfection : involves the addition of chemicals in order to kill or reduce the number of pathogenic organisms



During the treatment processed, colloids, suspended matter, pathogenic microorganisms and organic matter are removed to meet the regulated standard. However, the quality of raw water isn't always stable, and corresponding actions are required to be promptly adopted when events like the surge of pollutants or the large variability of the influent flow. In any event, the treated water from DWTPs should generate drinking water which complies the World Health Organization's Guidelines (WHO's guideline) for drinking water quality. Otherwise, the treated drinking water should either be discharged and result in the short term outage of water supply to the downstream cities or the users will receive contaminated drinking water which can potentially transmit diseases and cause illness.

%Turbidity
Turbidity is one of the critical water quality indicators, which can be defined as the "optical quality" of water, and the unit to decribe the turbidity is called Nephelometric Turbidity Unit (NTU). High levels of turbidity in raw water can impede the effectiveness of filtration and chlorination processes, and potentially cause short-term outages of water supply. Heavy rainfall and fissures within the aquifer can also lead to turbidity events are mostly likely to cause high turbidity \citep{worldhealthorganizationWaterQualityHealth2017}. The challenge in event of high turbidity in raw water is it occurs rapidly and mitigating activities must be actionable immediately. To address sudden event of such, \citet{stevensonAdvancedTurbidityPrediction2019} trained forecasting models based on general linear model (GLM) and RF to predict the time when the turbidiy reaches higher than 7 NTU. The results indicate both model can successfully predict the events (i.e., with accuracy between 0.81 and 0.86), and RF model is found to have higher precision due to it's ability to capture the nonlinear relationship between rainfall (mm) and turbidity (NTU).

%Coagulation dose
To maintain operational costs and water quality in the coagulation process, the amount of coagulant, which is mainly subject to the turbidity and alkalinity in the raw water, is traditionally determined thourgh manually sampling and analysis. Jar test is designed to find out the optimal chemical dosage for coagulation to remove the turbidity in raw water, and the entire process includes on-site sampling and up to more than 40 minutes of laboratory works \citep{ganiEffectPHAlum2017}. To replace the laborious procedure of jar tests, \citet{wangIntegratingWaterQuality2022} proposed using principal component regression (PCR), support vector regression (SVR), and long short-term memory (LSTM) neural network to build predictive models for outputing daily estimated chemical dosage. Compared with linear PCR model, nonlinear SVR and LSTM models captures more relationship between the chemical dose (e.g., ferric sulfate) and the raw water quality based on a higher R-squared value of 0.70.

%membrane-filtration modeling

%Disinfection
Disinfection is the last step of water treatment processes in drinking water treatment plants to generate safe potable water. In this step, one or more chemical disinfectants like chlorine, chloramine, or chlorine dioxide are added into the water to inactivate any remaining pathogenic microorganisms. However, the chlorination process requires precise dosing of disinfectant---too high will lead to the formation of disinfection byproducts (DBPs), and too low will result in insufficient levels of the residaul disinfectant concentration. In both senarios, the treated drinking water can pose health threats to the end users. The aforementioned PID controller can achieve automatic dosing of disinfection, however, \citet{wangModelPredictiveControl2020} found out that the accuracy of the predicted disinfectant dosage using (i.e., chlorine is used in this paper) a Support Vector Regression (SVR) model outperformed a PID controller in both simulation and experimental conditions. An Artificial Nerual Network based model also shows a more satisfied cost reduction in a chlorination dosing control system comapred to PID controller \citep{librantzArtificialNeuralNetworks2018}.

%DBPs
The invariability of the raw water quality is always a big issue for disinfection. For instance, chlorine dose can be excessive dosed when the treated water contains less pollutants (e.g., non-organic matters and ammonia nitrogen). Exessive addition of chlorine results in the problem of wasting chemicals which is reflected on the increase operational cost and potentially generate undesired disinfection by-products (e.g., trihalomethanes (THMs), which are carcinogenic to human) due to the chemcial reaction between pollutants and overly dosed chlorine. \citet{xuUsingSimpleEasy2022} trained an ANN model for predcting the occurrence of THMs in tap water using simple and easy water quality parameters (e.g., pH, temperature, $UVA_{254}$ and residual chlorine ($Cl_{2}$)). Despite the results showed a good model accuracy in predicting for THMs (i.e., T-THMs, TCM and BDCM), the applications of the model is largely limited in reality due to the lack of dataset regarding the quantity and quality . In fact, lack of high quality dataset for trianing ML models is a common issue, which explains up until recently, mathametical or empirical based AI models like fuzzy logic \citep{gamizFuzzyGainScheduling2020,godo-plaControlPrimaryDisinfection2021} is still widely used for process control in WTPs.

\subsection{Wastewater treatment plants}
%Background
Human activies produce wastewater and discharge from homes, businesses, factories and commercial activities to the sewage systems which connect to wastewater treatment plants (WWTPs). The function of a WWTP is to remove contaminants from sewage and water so that the treated water can be returned to the natural water body without dangering any living beings reside in the ecosystem. Undertreated wastewater can lead to harmful algal blooms or cause oxygen deficit in the water (i.e., low oxygen content can kill the fishes). The steps for treating municipal wastewater involve three major categories---primary treatment, secondary treatment and tertiary treatment. The pollutants which will either float or settle will be removed in primary treatment; next, secondary treatment is mainly responsible for removing BOD$_5$ in the biological processes; in the final tertiary treatment, membrane filtration, adsorption by activated carbon and addition of disinfectant can be applied optionally to futher eliminate the undesired pollutants in the water.

Wastewater can be defined as the flow of used water discharged from homes, businesses, industries, commercial activities and institutions which is transported to treatment plants via pulbich sewer system or engineered network of pipes. This wastewater is further categorized and defined according to its sources of origin. Domestic wastewater refers to water discharged from residential sources generated by kitchen wastewater, cleaning and personal hygiene. Industrial/commercial wastewater is generated and discharged from manufacturing and commercial activities, such as textile industry and food and beverage processing wastewater. Institutional wastewater characterizes wastewater generated by large institutions such as hospitals and educational facilities. Regardless of the source of the wastewater, WWTPs have to achieve at least three sustainability targets: environmental protection (i.e., low pollutants discharge), social acceptance (i.e., human sanitary protection) and economic development (i.e., feasible operational and management costs) \citep{manninaDecisionSupportSystems2019}. To effectively achieve these goals, process control is requried to reduce energy consumption, improve on effluent quality, and save costs in plant operation and management. The focus of this study is on discussing the development of using process control for treatment operation and management.

Before the deployment and use of DSS, the existing techniques for WWTP management showed several drawbacks: • Difficulties to manage the high complexity of WWTPs due to the interaction of heterogeneus components and elements (biological, chemical, physical, mechanical, etc.) • Lack of control, automation and instrumentation in WWTPs to cope with the dynamicity of WWTPs • No exhaustive alternative decision analysis support • No prognosis capabilities for possible alternative decision assessment • No wide data-based models use

%Prediction of effluent quality for evaluating treatment performance

Under known operational conditions of a WWTP, machine learning models can be trained to assist the plant operators optimize treatment processes to improve effluent quality . \citet{wangMachineLearningFramework2021} proposed a machine learning framework, utilizing a model based on Random Forest Regression to predict the effluent Total Suspended Solid (TSS) and phosphate (PO$_4$). The breakthrough achieved in this study features 

DSSs can be applied to WWTPs in order to predict the effluent quality under known WWTP operational conditions (Nadiri et al., 2018). For example, it is possible to adopt DSSs in view of calculating treatment efficiency and evaluating the removal of substances even prior to initiate the treatment under different operational conditions, implemented processes and influent features (Hamed et al., 2004; Sonaje and Berlekar, 2015).

Improve on effluent quality 
analyses was developed and used to model WWTP processes and investigate how operational variables influence effluent quality.

%prediction of polluants in treatment process

%disinfecion?

%BOD
The importance of BOD5 modeling stems from the extensive laboratory procedures performed to measure the BOD5 concentration, as these tests require approximately five days. 

The above soft sensor is expected to decrease the five-day period of the BOD5 measurement to several hours, which will allow the use of online control systems. \citep{alsulailiArtificialNeuralNetwork2021}

%Commonly targeted water paramters
%secondary
%%BOD5, COD, ammonia
%%Heavy metals
%%Orgainic pollutants

%tertiary
%%Disinfection
Disinfection in a water- and wastewater-treatment plant is the process by which microorganisms and viruses are killed or inactivated, mainly with chlorine-based disinfectants [64]. While chlorination is effective as a disinfectant, it also poses human health hazards [65]. Beyond its ability to cause acute toxicity in humans, chlorine is also known to interact with bromide and organic matter naturally found in water systems to form what is known as disinfection by-products. Disinfection by-products (DBPs) are suspected human carcinogens and reproductive disruptors, and have received increased scrutiny from regulators all over the world [66]. DBPs mainly belong to two larger subcategories: trihalomethanes (THMs) and haloacetic acids (HAAs). THMs are regarded as the most common form of DBPs as their formation is associated with chlorine disinfectants [67]. Haloacetic acids are commonly tested for five or nine common haloacetic acids and are commonly referred to as HAA5 or HAA9. The entire mechanism behind the formation of DBPs in drinking water is not known, making their prediction and mitigation an ideal candidate for ML technologies. When learning has been achieved, mitigation through control using AI methods is possible.
%%DBP




\subsection{Reclaimed water system and water body}
%Describe the need of using decision-making processes for water quality
%Using examples to show how the decision-making process can benefit water quality control
%%%%%%%%%%%%%%%%%%%%%%%%%%%%%%%%%%%%%%%%%%%%%%%%%%%%%%%%%%%%%%%%%%%%%%%%%%%%%%%%%%%%%%%%%%%%%%%
%Organization of the paragraph 
%What is a critical water parameter(turbidity for example), and then the difficulties in %%%%%%
%predicting the water parameter, for example, what are the related parameters to turbidity%%%%%
%%%%%%%%%%%%%%%%%%%%%%%%%%%%%%%%%%%%%%%%%%%%%%%%%%%%%%%%%%%%%%%%%%%%%%%%%%%%%%%%%%%%%%%%%%%%%%%

In this study the new control objectives for the reclaimed water system in Shek Wu Hui Effluent Polish Plant have been established: to monitor color and ammonia concentration in the MBR effluent and at the same time provide a predictive model to assist the disinfection control strategy for disinfecting the MBR effluent to meet the endorsed reclaimed water standard.

From many of the literatures proposing using AI models to build predictive models, we can observe that nonlinear models (e.g., Random Forest and Support Vector Regression) outperform the linear model (e.g., General Linear Model and Principal Component Regression). In 

Model Predictive Control (MPC) regardless of using AI models or ML models, both showed the superior performance over the PID controller. Despite the use of ML models has proved to outperformed the AI models in other industires, ML and AI models seem to be eqaully good in the process control in WTPs. Based on the senario, non-linear ML models don't always outperform linear AI models, and there is even a hybrid model (i.e., neuro-fuzzy inference systems (ANFIS)) developed to solve control issues in aerobic granular sludge reactors \citep{zaghloulDevelopmentEnsembleMachine2021}.
\section{Tools and techniques for enchancing the performance of machine learning modeling}
\subsection{Programming languages}
%why python? compared to matlab
%the library and support
\subsection{Data preprocessing}

\subsection{Feature engineering}
%what are the frameworks, and why using them

