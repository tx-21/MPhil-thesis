\chapter{Literature Review}
%\label{sec:ob_rel}

\section{The application of machine learning techniques for water quality control}
Machine learning is a subset of artificial intelligence, and deep learning is a subset 
of machine learning. In artificial intelligence can be sued to solve four types of problems: 
classification, regression, dimensionality reduction and clustering. 
\subsection{Water quality control in drinking water treatment plants}
%%it's better to find papers that can provide comparison between DL and ML or traditional models
\subsubsection{Membrane fouling}
Madfs
%membrane fouling
%membrane fouling focus more on the combined use of physical and ml models rather than single use.
%analysis of precursors of DBPs
%Disinfection
%coagulation/flocculation
%analysis and predictions of algae cells and algae-derived substances in source water
\subsubsection{Analysis of precursors of DBPs}
%Genetic algorithm (GA)\\
%Genetic programming (GP)\\

\subsubsection{Disinfection}
\subsubsection{Prediction of the source water contaminants}
\subsubsection{Coagulation}
Traditional modelling methods 
mainly use numerical simulations or physical formulas to model target prediction objects from 
a microscopic perspective . For example, the advantage of particle coagulation dynamics 
simulation is that it can ex    plain the behaviour evolution mechanism of particles in the water 
treatment process in a very specific way because it is usually based on the collision mechanism 
with physical meaning and mathematical description

\subsection{Water quality control in wastewater treatment plants}
\subsection{Water quality control in reclaimed water system}
%Describe the need of using decision-making processes for water quality
%Using examples to show how the decision-making process can benefit water quality control
%%%%%%%%%%%%%%%%%%%%%%%%%%%%%%%%%%%%%%%%%%%%%%%%%%%%%%%%%%%%%%%%%%%%%%%%%%%%%%%%%%%%%%%%%%%%%%%
%Organization of the paragraph 
%What is a critical water parameter(turbidity for example), and then the difficulties in %%%%%%
%predicting the water parameter, for example, what are the related parameters to turbidity%%%%%
%%%%%%%%%%%%%%%%%%%%%%%%%%%%%%%%%%%%%%%%%%%%%%%%%%%%%%%%%%%%%%%%%%%%%%%%%%%%%%%%%%%%%%%%%%%%%%%

In this study the new control objectives for the reclaimed water system in 
Shek Wu Hui Effluent Polish Plant have been established: to monitor color 
and ammonia concentration in the MBR effluent and at the same time provide 
a predictive model to assist the disinfection control strategy for disinfecting 
the MBR effluent to meet the endorsed reclaimed water standard.

\section{Recent advances in time series models for water quality forecasting}
\subsection{Machine learning models}
%why DL is better than ML?

\subsection{Deep learning models}
%why we need to use this?
%Explain the difference between DL and ML
\subsection{Comparison of the artificial intelligence model and traditional model in drinking water treatment}
\subsubsection{Traditional modeling methods}
In traditional modeling methods, numerical simulations or physical formulas to model target prediction objects
%The mathematical model or theoretical equation used was limited to the analysis and prediction of membrane pollution under simple conditions (such as a single pollutant or specific working conditions)
\section{Different techniques for enchancing the performance of forecasting models}
\subsection{Data preprocessing}
%why python? compared to matlab

\subsection{Feature engineering}
%what are the frameworks, and why using them

