\chapter{Literature Review}
%\label{sec:ob_rel}
\section{Introduction to water quality control}
\subsection{Automated control system for water quality control}
%%%%%%%%%%PLC and some real application cases in WTP
%Explain PLC better by \cite{WasteWaterTreatment2018}
%re organize this section
Programmable logic controller (PLC) is an industrial computer system designed for any process requiring a series of devices and equipment operates cohesively to achieve multiple purposes in manufacturing or treatment processes. The main components of PLC include a center process unit (CPU), input modules and output modules (I/O). CPU is responsible to process digital signals from input modules and send commands through output modules based on the control logics programmed on the PLC. For chemical dosing control in water treatment plants (WTPs), PLC system receives readings from turbidity and pH sensors and uses pumps to dose aluminum solution automatically \citep{andhareSCADAToolIncrease2014}. The PLC system with the capability of producing real-time output commands in response to the input sigals also makes it widely used in the wastewater treatment plants (WWTPs). For oxygen concentration control in the aeration tank, PLC system receives signals of dissovled oxygen (DO) detectors and transmits signals to open or close the electric butterfly valves to further alter the DO concentration \citep{zhuApplicationPLCSewage2017}. Although PLC systems are the most used system across industries for its easy programming and reliable control, PLC system is merely a device that can be programmed to control operative devices with on-off logic (i.e., a logic control with two states) and the capability of complex control is compromised. In reality, many WTPs or WWTPs have the need of precise control of the treatment processes. Being awared of the limitations of the PLC systems, a more advanced controller called proportional–interal–derivative (PID) controller for receiving analog signals was developed to obtain more sophisticated controls over the operative devices.

%%%%%%%%%%PID and examples
%PID control is used where greater levels of precision in control are required. It combines three control terms to give a single output to drive the setpoint. 
To react to rapidly-changing process conditions, a PID controller generates an output value based on continuous calculation of an error value e(t) as the difference between a desired setpoint (SP) and a measured process variable and applies a correction based on proportional, integral, and derivative terms. The use of the "P", "I", and "D" allows the system to quickly reach steady state with a feedback control system (i.e., the system output is returned to the system input which is included in the decision making process in PID controller). Generally speaking, a PID controller is a technology (i.e., a specialist algorithm) for contorlling a single device with more complex logics, while a PLC system is a physical system consists of different modules and capable of controlling dozens of devices only with two-state logic. In addition, A PID controller can be designed to operate on PLC device and provide a more precise control strategy to a designated device. In WWTPs, a single-variable feedback analog control loop in PID can be used to control the temperature in the activated sludge treatment by stablizing the system temperature in a shorter time \citep{badosDesignPIDControl2020}. The feedback control scheme is also applied in WTPs to adjust the addition of chlorine dosage (i.e., also known as the disinfection process, chlorination, or postchlorination) to reach the target concentration of free chlorine residual (FRC) \citep{wangCompositeControlPostChlorine2019}. Disinfection process is carried out in a chlorine contact tank which provides sufficient time for chlorine to disinfect pollutants. Since the chlorine added by the dosing device requires time to travel from the entry to the exit, the system output can only reflect the changes of water quality in a delayed time of 30 minutes (i.e., the designed time for water to travel in chlorine contact tank is usually 30 minutes or longer). In the case of chlorination, the lag of time makes feedback control difficult \citep{kobylinskiLineControlStrategies2006} as the system is delayed in responding to any sudden surge of the pollutants when it can only receive output at the end of the disinfection process. PID controllers in WWTPs also encounter similar challenges as the increasing complexity of water quality and stricter regulations on the discharged water quality. 

%%%%%%%%%%Transition to how the algorithms in PID can be replace by AI, ML, 
%Introduce to MPC, feedforward control in comparison with PID controller
%The use of mathematical modeling is not ideal, turning to the use of AI modeling
To tackle the difficulties encountered in process control system, many control strategies are proposed, such as feed forward-feedback control, linearized and optimal control, model-predictive control, and fuzzy control, etc \citep{demirFeedbackControlChlorine2014a}. Among the algorithms used in control strategies, Artificial Intelligence (AI) modeling has gained the most attentions in recent years compared to modeling based on mathematical models or empirical formulas. In WTPs or WWTPs, to fully understand the physical, biological, and chemical interactions in the treatment plants is very difficult. The unpredictable behaviors during the water treatment can be the significant changes of influent flow rate, flucutations of water quality, the complexity of biological treatment process, and the large time delay exists between this control variable and the process input, etc. Therefore, AI modeling shows a great potential in dealing with the highly complex conditions in the treatment process \citet{liRecentAdvancesArtificial2021}. In the next sections, the applications of different AI modeling methods will be discussed.

\subsection{Artificial Intelligence}
%AI include fuzzy logic, a subset of ML. ML includes SVM, a subset of DL. DL includes LSTM.
Artificial intelligence (AI) can perform cognitive tasks with the development of computational solutions. The concepts of AI are usually confused, in fact, AI is a very broad term and any kind of algorithms or models which involved in decision-making with computation fall in the domain of AI. For example, fuzzy logic and optimization algorithm are formulated with human design and computer decision making process. There are another subset of AI called machine learning (ML), but the process of generating a ML model is different to generating a fuzzy logic model. ML uses learning algorithms to generate a model via learning from historical or large amount of data without being explicitly programmed. ML algorithms can be classified into three categories, which are Supervised, Unsupervised, and Reinforcement learning. In the training process of supervised learning, input variable (x) and output variable(Y) we will provided, and model will learn from the provided dataset to map the x to the Y. A trained supervised model can generate a prediction for the response to the new data (i.e., also called the unseen data). Unsupervised learning is when the dataset is not labelled, the model can learn to infer patterns in the dataset without reference to the known outputs. This type of algorithm can find similarities and differences in the data. In reinforcement learning, models are designed to constantly interact with the environment in a try-and-error way and recieved rewards and punishments based on the purpose of the tasks. Generating a optimal action to achieve lowest penalties is the main function of a reinforcement learning model. In process control, supervised learning are frequently used in many senarios.

Regression is a supervised machine learning technique used to predict continuous values. A regression model can estimate the relationship between the input variables in the system and the output target from a given dataset, and then use the nonlinear relationship to map the unseen input data to a predicted output data. This type of application is sutiable for water quality prediction  \citep{librantzArtificialNeuralNetworks2018}, and sensor fault detection \citep{cecconiSoftSensingOnLine2021}, etc. 

\subsection{AI modelings for water quality control}
In machine learning, there are some popular models used by the researchers. Supporting vector machine (SVM), random forest (RF), and Artificial Neural Networks (ANN) are used to train predictive model. \citet{librantzArtificialNeuralNetworks2018} trained a RF model to predict the free residual chlorine concentration (FRC) in a WTP, and \citet{xuAlternativeLaboratoryTesting2021} built a RF-based model to predict total nitrogen concentration in water bodies. \citet{guoPredictionEffluentConcentration2015} compared the reliability and accuracy of an ANN model and a SVM model in predicting 1-day interval T-N concentration in a WWTP, and the results showed that RF model has higher accuracy while ANN model is more reliable for assisting decision-making process. Fuzzy logic approach


RF is commonly used for building prediction models, for instance, \citet{xuAlternativeLaboratoryTesting2021} uses RF-based model to predict total nitrogen concentrations in water bodies, and \citet{bulacanstateuniversitycityofmalolosbulacanphilippines.heisalsowithamauniversityquezoncityphilippinesPredictingRiverPollution2020} used the RF model to predict the water pollution level. ANN can also perform predicting tasks, such as predicting the free residual chlorine concentration (FRC) in a WTP \citep{librantzArtificialNeuralNetworks2018}. 


As the the computing power doubled every 18 months according to Moore's law. A subset of ML, Deep Learning (DL) becomes more accessible for sovling everyday issues. In simplicity, DL models can be defined as neural networks with more than two hidden layers (i.e., the model complexity increased and required more computing power to calculate). In DL, there are various types of architectures designed based on the type of problems. For image processing, Convolutional Neural Network (CNN) is designed to extract important features from the image vectors. Another popular DL architecture is Recurrent Neural Network (RNN), which is powerful in solving time series-related applications and Natural Language Processing (NLP) tasks \citep{liERNNDesignOptimization2018}. Although each architecture has their strengh in tackling different types of problems, both architectures can be used for a single task \citet{liPredictionFlowBased2022} built a regression CNN-RNN model for rainfall-runoff prediction. DL can be extremely powerful when multiple architectures ared fused into a single model to perform a specific task, which cannot be realized by machine leraning models. That being said, DL can achieve higher model performance in terms of the prediction accuracy compared to ML. 

%Explain RNN

%Explain CNN

\section{Water quality control with the use of machine learning modeling}
%%it's better to find papers that can provide comparison between DL and ML or traditional models
\subsection{Drinking water treatment plants}

Disinfection is the last step of water treatment processes in drinking water treatment plants (DWTPs) to generate safe potable water. In this step, one or more chemical disinfectants like chlorine, chloramine, or chlorine dioxide are added into the water to inactivate any remaining pathogenic microorganisms. The residual disinfectant concentration in disinfected water must contains low levels of the chemical disinfectant to stop nuisance growths in the water distribution pipes, storage facilites and conduits. Nowadays, the widely used disinfectant in disinfection process is chlorine as gas or hypochlorite (i.e., in form of liquid solution), and the treatment process is known as "chlorination". According to World Health Organization's Guidelines for Drinking-water Quality (WHO Guidelines), the maximum allowable value for free chlorine residual in drinking water is 5 mg/L, and the minimum recommended valus is 0.2 mg/L. 

Current analysis proposes a multivariable control for post-chlorination dosage system in a WTP using artificial neural networks applied to the disinfection process to reduce free residual chlorine variations of treated water in the water tank and, consequently, in the main water distribution \citep{librantzArtificialNeuralNetworks2018}.

%1-Why there is chlorinatino process
%2-The automation process in chlorination to control the conc. of residual chlorine
Despite the benefit brought by dosing chlorine to the water, negative impacts also come along. In the real world case, the influent water quality and the efficiency of the drinking water treatment processes are not always stable, and the invariability of the treated water quality becomes an big issue for disinfection. For instance, chlorine dose can be excessive dosed when the treated water contains less pollutants (e.g., non-organic matters and ammonia nitrogen). Although the quality of disinfected water fulfill the regulation standard, it increases the costs and can potentially generate undesired disinfection by-products (e.g., trihalomethanes, which are carcinogenic to human) due to the chemcial reaction between pollutants and overly dosed chlorine. On the flip side, insufficient dosing of chlorine cuases the concentration of residual chlorine lower than the legal regulation. To prevent both senarios occur, a water quality control strategy is required to produce drinking water with satisfactory quality. 

Up untill present, there are several ways to perform disinfected water quality. In the earliest time, feed-back.... PI... feed-foward...
%Chlorine Chlorine is one of the most commonly used disinfectants for water disinfection
%Chlorination in drinking water treatment plants (DWTP) is the final process applied to water before it is sent to storage tanks in the supply network for subsequent human consumption. An excessive dosage of chlorine or, conversely, too small a dosage, may breach existing legal regulations on mandatory limits. Furthermore, excessive amounts generate an unnecessary cost in terms of chlorine and, collaterally, problems due to an increase in the maximum permitted amounts of such by-products as trihalomethanes, which are carcinogenic compounds for humans. In DWTP where there is no significant variability in the quality of the water to be treated, a type of control that is proportional to the flow rate in the effluent can have fully satisfactory results. However, in a control strategy applied when there are inherently long delays in the process, variability in the quality of the water to be treated and considerable variations in flow, a proportional type of control does not tend to work and an alternative type is needed. This article presents the strategy and results of a control method that combines a feed-forward system with gain scheduling in a (ProportionalIntegral) PI control. The control system design was validated beforehand by simulation and then applied to a real DWTP, producing satisfactory experimental results.\cite{gamiz_fuzzy_2020}


\subsubsection{Membrane fouling}
Madfs
%membrane fouling
%membrane fouling focus more on the combined use of physical and ml models rather than single use.
%analysis of precursors of DBPs
%Disinfection
%coagulation/flocculation
%analysis and predictions of algae cells and algae-derived substances in source water
\subsubsection{Analysis of precursors of DBPs}
%Genetic algorithm (GA)\\
%Genetic programming (GP)\\

\subsubsection{Disinfection}
\subsubsection{Prediction of the source water contaminants}
\subsubsection{Coagulation}
Traditional modelling methods 
mainly use numerical simulations or physical formulas to model target prediction objects from 
a microscopic perspective . For example, the advantage of particle coagulation dynamics 
simulation is that it can ex    plain the behaviour evolution mechanism of particles in the water 
treatment process in a very specific way because it is usually based on the collision mechanism 
with physical meaning and mathematical description

\subsection{Wastewater treatment plants}
\subsection{Reclaimed water system and water body}
%Describe the need of using decision-making processes for water quality
%Using examples to show how the decision-making process can benefit water quality control
%%%%%%%%%%%%%%%%%%%%%%%%%%%%%%%%%%%%%%%%%%%%%%%%%%%%%%%%%%%%%%%%%%%%%%%%%%%%%%%%%%%%%%%%%%%%%%%
%Organization of the paragraph 
%What is a critical water parameter(turbidity for example), and then the difficulties in %%%%%%
%predicting the water parameter, for example, what are the related parameters to turbidity%%%%%
%%%%%%%%%%%%%%%%%%%%%%%%%%%%%%%%%%%%%%%%%%%%%%%%%%%%%%%%%%%%%%%%%%%%%%%%%%%%%%%%%%%%%%%%%%%%%%%

In this study the new control objectives for the reclaimed water system in 
Shek Wu Hui Effluent Polish Plant have been established: to monitor color 
and ammonia concentration in the MBR effluent and at the same time provide 
a predictive model to assist the disinfection control strategy for disinfecting 
the MBR effluent to meet the endorsed reclaimed water standard.


\section{Tools and techniques for enchancing the performance of machine learning modeling}
\subsection{Programming languages}
%why python? compared to matlab
%the library and support
\subsection{Data preprocessing}

\subsection{Feature engineering}
%what are the frameworks, and why using them

