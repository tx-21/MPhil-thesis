\chapter{Introduction}

\section{Background}

%paragraph 1 Urban water scarcity-why is it an issue
%paragraph 2 Water reclamation-why is it a good choice for solving urban water scarcity
%paragraph 3 Decision-making processing-how does this help water reclamation
%paragraph 4 Deep learning model to replace fuzzy supervisor and machine learning models-the need of using it
Intelligent robots are influencing many aspects of the world nowadays, from collaborative robot arms in factories to L4 autonomous driving technology, from biped household robots to quadruped mobile military agents, and from unmanned surface vehicles to quadrotor swarms. The more deeply we imagine the future, the more indispensable we find robot services.

For mobile robots, navigation is always the kernel function. However, compared with the automation of manipulation, mobile robot navigation is evolving more slowly. In the classical manipulation and manufacturing scenarios, the workspace of robots is usually well defined, and the robots can perform correctly under human-designed programming, without any interaction between uncertain objects and themselves. If we regard \textit{replacing the repetitive workloads} as the first step, we should start to consider \textit{living with the robot} in the next step. For mobile robots, to be incorporated into human's daily life, they must be intelligent in unknown and pedestrian-rich environments for tasks like autonomous driving, cargo delivers and household mobile robots, which is \textit{behaving like a human}.

Human beings can navigate in crowded environments smartly without dependence on pre-defined high-resolution maps. We can also explore unknown environments without taking time to think about the information gain or frontiers. To navigate safely and efficiently like a human being, mobile robots should be able to perceive and predict behaviours of unfamiliar and dynamic agents. Based on the implicit or explicit understanding, an accessible policy considering various constraints is needed to guide the robots.


Deep learning, as a solution for artificial intelligence, is capable of building progressively meaningful feature abstraction of input data.
It plays an essential role in various fields of study bringing the state of the art in image classification 
semantic segmentation ,
human-level game playing , driving real robotic systems in navigation 
and manipulation  tasks.

We may be witnessing the most rapidly growing trend of deep learning techniques for robotics tasks in recent years.
Replacing hand-crafted features with learned hierarchical distributed deep features, and learning control policies directly from high-dimensional sensory inputs, the robotics community is making substantial progress towards building fully autonomous intelligent systems.Bla.\cite{chen_assessing_2003}.

\section{Objectives}
\noindent
The specific objectives of this thesis work are:\\
%should be investigate the effluent water quality in SWHEPP?
(1) To investigate the daily patterns of Color and Ammonia concentration in SWHEPP.\\
(2) To build deep learning models for training Color and NH$_{3}$N forecasting models.\\
(3) To develop methods of data pre-processing and feature engineering for training forecasting models.\\
(4) To elucidate the architecture design of the deep neural networks.

\section{Organization of the thesis}