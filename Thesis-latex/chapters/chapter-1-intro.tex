\chapter{Introduction}
\section{Background}
%stress out the importance of using reusable water
Urban water challenge increases as the cities grow larger. The World Bank estimates that the urban population worldwide will double by 2050---with severe implications of escalating water demands in cities by 50--70 percent \citep{theworldbankCircularEconomyOpportunity2021}. Global climate change has primarily affected the amount, distribution, and quality of the available fresh water in the urban water cycle. The report from \citep{unicefURBANWATERSCARCITY2021} points out that one in four cities is facing challenges in supplying adequate water to inhabitants, and the situation is even worse in cities in the developing world. The rise of urban water usage will generate more wastewater. Thus, converting municipal/industrial wastewater into reusable water has recently drawn much attention. Reuse water increases availability by substituting freshwater for non-potable (drinkable) uses for agricultural irrigation, industrial and urban water reuse, etc. The alternative reuse water can supply many activities and save drinking water for other purposes elsewhere \citep{adewumiTreatedWastewaterReuse2010}.

%
The construction of reclaimed water facilities often requires a huge amount of capital investment. Upgrading available wastewater treatment plants with reuse water treatment facilities is an economical solution accompanied by the potential of realizing resource recovery (e.g., nitrogen and phosphorus recovery) \citep{maryamWastewaterReclamationReuse2019,kehreinCriticalReviewResource2020}. The primary concern of reusing treated wastewater is the potential risks caused to public health. Under unexpected circumstances, the reclaimed water facilities can produce unqualified reclaimed water, which is harmful to the living beings (i.e., as reused water is ingested directly or through irrigated crops) and irrigated soil \citep{adewumiTreatedWastewaterReuse2010}. In Hong Kong, reclaimed water quality is regulated with up to 10 or more water quality parameters, and any parameters that fail to meet the standard will lead to disqualification. The common practice for controlling the treated water quality is achieved through water quality control strategies. The market controllers have evolved from a simple on-off logic controller called Programmable Logic Controller (PLC), to a more advanced multi-step response controller called proportional-integral-derivative (PID), and finally to the controller consists of machine learning models.

%controller can refer to the entire system or the mathematical equations
The uses of machine learning models in the water quality controllers for assisting water quality control strategy are ground-breaking applications. Many research papers have proposed various machine learning models for replacing the PLC and PID controllers and demonstrated the benefits of machine learning models. From the study of \citep{librantzArtificialNeuralNetworks2018}, PID and machine learning-based controllers were deployed to compare the operational costs of dosing the chlorine to the setpoint concentration in a drinking water treatment plant. The results showed that the Artificial Neural Network-based model has a more satisfactory cost reduction in a chlorination dosing control system than the PID controller. Another research finding suggests using a Support Vector Regression (SVR) model as the controller required less time to reach the setpoint concentration of free chlorine residual compared to the PID controller in both simulation and experimental conditions \citet{wangModelPredictiveControl2020}. Incorporating machine learning models in traditional process control systems has also been practiced by \citet{santinFuzzyControlModel2015} for avoiding violations of total nitrogen in the effluent using the decisions made by Artificial Neural Networks. Long Short-term Model was also used to predict which process control strategy should be selected for eliminating violations of total nitrogen concentrations in the effluent \cite{pisaLSTMBasedWastewaterTreatment2019}. Forecasting water quality or predicting future events using machine learning are proved to be effective measures for controlling effluent water quality in wastewater treatment plants, making these approaches to be promising solutions for the reclaimed water treatment operation and management.

The superior performance of machine learning models comes from training high-quality datasets with a good amount of data that can fairly represent the system's dynamic. Most studies have only focused on evaluating the model performance by comparing the test loss values between models and the improvements over PID controllers without considering the collected dataset's quality. The noises in the data and the number of features (i.e., inputs or variables) are the two critical factors affecting machine learning models' accuracy and robustness. Many data pre-processing techniques are proposed and applied to enhance the dataset's quality. For instance, some papers discussed pre-processing data for removing the noise in raw datasets using data smoothing filters \citep{chengForecastingWastewaterTreatment2020}, or creating new features in addition to the original ones \citep{mamandipoorMonitoringDetectingFaults2020} to achieve data augmentation. Despite the efforts being made, the influences of the proposed data pre-processing techniques on the final model performance have not yet been established.

Machine learning models for water quality control have two main types of algorithms, regression and classification. The former provides forecasting results of specific values, while the latter provides a decision of yes or no (i.e., 1 or 0). The regression model is also called the forecasting model, which plays a vital role in water quality control in drinking water treatment plants (DTPs) 
and wastewater treatment plants (WWTPs) by forecasting the future water quality. The need to use forecasting models is due to the unpredictable nature of water quality. Yet, the treatment operations are required to produce effluent satisfying the government regulation \cite{chenAssessingWastewaterReclamation2003} regardless of how the influent water quality may vary daily. In the reclaimed water system in Shek Wu Hui Effluent Polish Plant (SWHEPP), forecasting models are recommended for effluent treatment management and operation. However, only limited online sensors are available onsite. Despite the limited data for model training, it is still possible to train forecasting models with one feature, which is known as the univariate forecasting model. In this study, we will attempt to build machine learning models for forecasting water quality parameters in reclaimed water. Meanwhile, data pre-processing methods will be proposed and evaluated to address the research gap of insufficient understanding of data pre-processing in modeling in the wastewater treatment industry.

%6. Explaining the key terminology in your field

%Explaining how you will use terminology and acronyms in your paper

%paragraph 1 Forecasting models play an important role in water quality control in DTPs and WWTPs.
%paragraph 2 Water reclamation-why is it a good choice for solving urban water scarcity
%paragraph 3 Decision-making processing-how does this help water reclamation
%paragraph 4 Deep learning model to replace fuzzy supervisor and machine learning models-the need of using it
%AI technologies have been successfully applied to different DWT processes, such as the prediction of the coagulant 
%dosage, discrimination of the DBP formation potential, advanced control of membrane fouling, membrane preparation 
%and optimization, and water quality prediction. \cite{liRecentAdvancesArtificial2021}

\section{Objectives}
\noindent
The specific objectives of this thesis work are:\\
%should be investigate the effluent water quality in SWHEPP?
(1) To build baseline univariate forecasting models using machine learning and deep learning models.\\
(2) To develop data pre-processing techniques for removing data noise for enhancing model performance.\\
(3) To extract relevant information from reclaimed water system using domain knowledge for feature engineering.\\
(4) To create new features for augmenting dataset's quality for further improving forecasting model performance.

\section{Organization of the thesis}
In Chapter 1, “Introduction,” the background information, objectives, and organization of the thesis were presented.

Chapter 2, “Literature Review”, provides an overview of water quality control strategies in water treatment plants, wastewater treatment plants, and reclaimed water systems.

In Chapter 3, “Materials and Methods,” the instruments for data collection of ammonia concentrations and colour levels, computer programming environment, and data preparation techniques were summarized. The processes of formulating extra features for training forecasting models were illustrated.

In Chapter 4, “Results and discussion,” the performance of machine learning and deep learning models were compared. Forecasting models trained by different data pre-processing techniques and the influences of feature engineering on model performance were compared with the baseline model performance in test loss. 

In Chapter 5, “Conclusions and Recommendations,” the findings obtained from this thesis work were summarized, and possible future studies were recommended.