\chapter{Conclusions and Recommendation}
\section{Conclusions}
\subsection{Machine learning models vs deep learning models}
The selection of using which machine learning and deep learning models was not widely discussed to the best of our knowledge in the modeling of the wastewater treatment industry. This study has investigated the model performance of the machine learning model of RF and four other deep learning models of DNN, RNN, GRU, and LSTM on forecasting ammonia concentrations and colour levels in the reclaimed water system for assisting treatment operation and management. The evidence from this study suggests deep learning models are much capable of learning from historical data and generating more accurate forecasting results. In both ammonia and colour forecasting models, the test loss values of RF are much higher than those of the least-performance deep learning model of DNN. Among all the deep learning models, the results indicate that LSTM and GRU models have the lowest test loss of 0.0405 and 0.0414, respectively. However, further research works suggest that LSTM models trained with pre-processing methods generate the lowest test loss compared to GRU, making the LSTM model the most promising recurrent neural network model for training forecasting models.

\subsection{Data pre-processing techniques}
Our research also highlighted how the model performance could be improved by applying data pre-processing and feature engineering techniques. Generally speaking, all the proposed data smoothing and outlier removal methods reduced the test loss values compared to the baseline model performance (i.e., the window sizes of the filters need to be carefully selected), as shown in Fig.~\ref{fig:preprocessing-comparison}. It is believed that the convoluted data points generated from data smoothing filters enable the recurrent neural networks to predict future values more easily.

%This paper has investigated on how much the data pre-processed methods and feature engineering techniques can improve the performance on the ammonia and colour forecasting models. The evidence from this study suggests ammonia forecasting model trained by SG filter (i.e., LSTM-1-sg7) reduced test loss by 4.2\%, and model trained with engineered features (i.e., LSTM-4-sg7) reduced test loss by 8.9\%. Showing 

\subsection{Feature engineering}
This study is the first step towards enhancing our understanding of the potential benefits of using created features for model training. The thorough examinations of the Geomap near the SHWEPP and the investigation of water composition in the public sewage system helped us hypothesize that the change of ammonia concentrations and colour levels depend on each other. With the help of additional colour/ammonia data for the ammonia/colour forecasting models, the test loss was reduced by 6.4\% (i.e., LSTM-2-sg7 compared to LSTM-1-obs) and 10.8\% (i.e., LSTM-2-ew4 compared to LSTM-1-obs), respectively. 

Moreover, the similarity between the household consumption patterns and the daily fluctuation of ammonia concentrations have unexpectedly helped us to formulate the time features via positional encoding. The influence of the sine and cosine hour features on the model performance showed tremendous improvements in both ammonia and colour forecasting models. In the former, test loss dropped by 8.9\% (i.e., LSTM-1-obs compared with LSTM-4-sg7) while the latter reduced by 28.6\% (i.e., LSTM-1-obs compared with LSTM-3-sg9). The remarkable use of positional encoding features is that they are not limited to ammonia and colour forecasting models. Any time-series data characterized by daily fluctuation patterns can adopt the use of the features of sine and cosine hour as long as the patterns are based on actual events. In addition, the positional encoding features are not limited to the hour component, we can encode time component features from seconds to weeks and even years, the application of it is infinite. However, the feature engineering method has some limitations. In the results of ammonia forecasting models, LSTM-2-obs, LSTM-3-obs, and LSTM-4-obs showed higher test loss compared to LSTM-1-obs, indicating that when the models were trained by any features other than ammonia, the model performance worsened. Another scenario from the colour forecasting results is that the test loss increases when the models are input with more than one feature and trained with EWMA filtered datasets. Our results suggest that feature engineering needs to be carefully evaluated and experimented with before its real application. Despite the limitations, the combination use of feature engineering in building ammonia and colour forecasting models in this study has fully proved its advantages. 

\section{Recommendations for future research}
Due to the insufficient amount of ammonia and colour data, we cannot differentiate whether the undesired model performance was caused by the heterogeneity of the training and testing datasets or caused by the pre-processing and feature engineering techniquess we applied to the datasets. It is recommended a larger dataset (e.g., a dataset with the size of more than two months) should be used in the future study when evaluating the proposed methods in this study. The insufficient data could also lead to the unstable performance of different models trained by the same data smoothing technques. For instance, LSTM-4-sg7 and LSTM-3-sg7 have the lowest test loss among LSTM-4 and LSTM-3 models; however, LSTM-2-ew4 has a lower test loss than LSTM-4-sg7. We failed to explain what has caused such an outcome. 

All the forecasting models in this study only focus on predicting ammonia concentration and colour levels, and in futher study, more water quality parameters should be included. In a reclaimed water system, the concentration of water quality parameters such as turbidity and E. coli are also regulated by Water Supply Department. Violating any water quality parameter will directly lead to the disqualification of being used as reclaimed water. Moreover, The hidden correlations that reside between each water quality parameter will most likely help build any water quality, forecasting models.

