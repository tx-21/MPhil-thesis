\chapter{Conclusions and Recommendation}
\section{Conclusions}
\subsection{Machine learning models vs deep learning models}
The selection of using which machine learning and deep learning models was not widely discussed to the best of our knowledge in the modelling of wastewater treatment industry. This study has investigated on the model performance of machine learning model of RF, and four other deep learning models of DNN, RNN, GRU and LSTM on forecasting ammonia concentrations and colour levels in reclaimed water system for assisting treatment operation and management. The evidence from this study suggests deep learning models are much capable in learning from the historical data and generate more accurate forecasting results. In both ammonia and colour forecasting models, the test loss values of RF are much higher than the values from the least performanced deep learning model of DNN. Among all the deep learning models, the results indicate LSTM and GRU models have the lowest test loss of 0.0405 and 0.0414, respectively. However, the further research works suggest LSTM models trained with pre-processing methods generate the lowest test loss compared to GRU, making the LSTM model the most promising recurrent neural network model for training forecasting models.

\subsection{Data pre-processing techniques}
Our research also highlighted the importance of how the model performance can be improved from applying data pre-processing and feature engineering techniques. Generally speaking, all the proposed data smoothing and outlier removal methods reduced the test loss values compared to the baseline model performance (i.e., the window sizes of the filters need to be carefully selected), as showin in Fig.~\ref{fig:preprocessing-comparison}. It is believed that the convoluted datapoints generated from data smoothing filters enable the recurrent neural networks to predict the future values more easily.

%This paper has investigated on how much the data pre-processed methods and feature engineering techniques can improve the performance on the ammonia and colour forecasting models. The evidence from this study suggests ammonia forecasting model trained by SG filter (i.e., LSTM-1-sg7) reduced test loss by 4.2\%, and model trained with engineered features (i.e., LSTM-4-sg7) reduced test loss by 8.9\%. Showing 

\subsection{Feature engineering}
This study is the first step towards enhancing our understanding to the potential benefits of using created features for model training. The thorough examinations of the Geomap nearby the SHWEPP and the investigation of water composition in the public sewege system helped us to hypothesize that the change of ammonia concentrations and colour levels are dependent to each other. With the help of additional colour/ammonia data for ammonia/colour forecasting model, the test loss reduced by 6.4\% (i.e., LSTM-2-sg7 compared to LSTM-1-obs) and 10.8\% (i.e., LSTM-2-ew4 compared to LSTM-1-obs), respectively. 

Moreover, the similarity between the household consumption patterns and the daily fluctuation of ammonia concentrations have unexpectedly helped us to formulate the time features via positional encoding. The influence of the sine and cosine hour features on the model perfromance showed tremendous improvements on both ammonia and colour forecasting models. In the former, test loss dropped by 8.9\% (i.e., LSTM-1-obs compared with LSTM-4-sg7) while the latter reduced by 28.6\% (i.e., LSTM-1-obs compared with LSTM-3-sg9). The remarkable use of positional encoding features is it is not limited to just ammonia and colour forecasting models. Any time series data characterized with daily fluctuation patterns can adopt the use of the features of sine and cosine hour as long as the patterns are based on actaul events. In addition, the positional encoding features are not limited to hour, we can encode features into from seconds to weeks, and even years, the application of it is infinite. However, the feature engineering method clearly has some limitations. In the results of ammonia forecasting models, LSTM-2-obs, LSTM-3-obs and LSTM-4-obs showed higher tess loss compared to LSTM-1-obs, indicating when the models were trained by any features other than ammonia, the model performance worsened. In addition to that, when extra features were train with EWMA filters, the test loss increased, and the worsening of model performance also occured on colour forecasting models trained by EWMA filters. Our results suggest that feature engineering needs to be carefully evaluated and experimented before the real use. Despite the limitations, the combination use of feature engineering in building ammonia and colour forecasting models in this study has fully proved it's advantages. 

\section{Recommendations for future research}
Due to the insufficient ammonia and colour data, we cannot differentiate whether the undesired model performance was caused by the heterogeneity of the training and testing datasets or caused by the pre-processing and feature engineering methods we applied on the datasets. It is recommended a larger dataset (e.g., dataset with the size of 2 months or longer) should be used in the future study when evaluating the proposed methods in this study. The insufficient amount of data could also lead to the unstable performance of different models trained by the same data smoothing methods. For instance, LSTM-4-sg7 and LSTM-3-sg7 have the lowest test loss among LSTM-4 and LSTM-3 models, however, LSTM-2-ew4 has the lower test loss than LSTM-4-sg7. We failed to explain what has caused such outcome. 

All the forecasting models in this study only focus on predicting ammonia concentration and colour levles, and in the futher study, more water quality parameters should be included. In reclaimed water system, the concentration of water quality parameters such as turbidity and E. coli are also regulated by Water Supply Department. The violation of any water quality parameter will directly lead to the disqualification of being used as reclaimed water. Moreover, The hidden correlations reside between each water quality parameter will most likely be helpful in building any water quality forecasting models.

