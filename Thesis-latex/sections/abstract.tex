\begin{abstract}

Water scarcity is a global challenge. One of the promising ways to mitigate the 
water resource crisis is via wastewater reclamation. Chlorine is commonly used 
for reclaimed water disinfection and requires precise dosing to satisfy endorsed 
quality standards. Ammoniacal nitrogen (NH$_{3}$N) and colour exist in the reclaimed 
water at concentrations between 0.23 – 5.44 mg N/L and 80 – 150 Hazen units, 
respectively, and can affect the chlorine demand. Forecasting the reclaimed water 
quality enables a feedback control system over the disinfection process by predicting 
the exact chlorine dose required which secures sufficient time to respond to sudden 
surges in color and ammonia levels. This study developed time-variant models based on 
machine learning to predict the NH$_{3}$N concentration and colour three hours into the 
future in the reclaimed water. The NH$_{3}$N data was collected by an online analyzer, and 
colour data was collected by a customized auto-sampling spectrophotometer, both are 
installed in the reclaimed water treatment plant in Hong Kong. Long Short-Term Memory 
(LSTM) was found to be the most effective architecture for training NH$_{3}$N and colour 
forecasting models. In the training processes, we applied data pre-processing methods 
and feature engineering, a technique to select or create relevant variables in raw data 
to enhance predictive model performance. From feature engineering, we discovered that the 
daily fluctuation in NH$_{3}$N and colour has correlations with the urban water consumption patterns. 
This finding further enhanced the NH$_{3}$N and colour forecasting model performance by 4.9\% 
and 5.4\% compared to baseline models. This research work offers novel methods and feature 
engineering processes for NH$_{3}$N concentration and colour forecasting in reclaimed water 
for treatment optimization. 

\end{abstract}
