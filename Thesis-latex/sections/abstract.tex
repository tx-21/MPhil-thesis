\begin{abstract}

%Water scarcity is a global challenge. One of the promising ways to mitigate the water resource crisis is via wastewater reclamation. Chlorine is commonly used for reclaimed water disinfection and requires precise dosing to satisfy endorsed quality standards. Ammoniacal nitrogen (NH$_{3}$N) and colour exist in the reclaimed water at concentrations between 0.23 – 5.44 mg N/L and 80 – 150 Hazen units, respectively, and can affect the chlorine demand. Forecasting the reclaimed water quality enables a feedback control system over the disinfection process by predicting the exact chlorine dose required which secures sufficient time to respond to sudden surges in color and ammonia levels. This study developed time-variant models based on machine learning to predict the NH$_{3}$N concentration and colour three hours into the future in the reclaimed water. The NH$_{3}$N data was collected by an online analyzer, and colour data was collected by a customized auto-sampling spectrophotometer, both are installed in the reclaimed water treatment plant in Hong Kong. Long Short-Term Memory (LSTM) was found to be the most effective architecture for training NH$_{3}$N and colour forecasting models. In the training processes, we applied data pre-processing methods and feature engineering, a technique to select or create relevant variables in raw data to enhance predictive model performance. From feature engineering, we discovered that the daily fluctuation in NH$_{3}$N and colour has correlations with the urban water consumption patterns. This finding further enhanced the NH$_{3}$N and colour forecasting model performance by 4.9\% and 5.4\% compared to baseline models. This research work offers novel methods and feature engineering processes for NH$_{3}$N concentration and colour forecasting in reclaimed water for treatment optimization. 

Water scarcity is a global challenge, and one of the promising ways to mitigate the water resource crisis is via wastewater reclamation. Water quality and aethetics are the primary concern in reclaimed water since undertreated water can pose health risks and unpleasant colour are likely to induce public misgiving. Ammoniacal nitrogen (NH$_{3}$-N) and colour substances exist in the reclaimed water and can severly affect the reclaimed water quality in different ways. Chlorine is commonly used for reclaimed water disinfection and requires precise dosing to satisfy endorsed quality standards, however, NH$_{3}$-N consumes chlorine and affects the chlorine dosing. Colour substances do not consume chlorine, but it requires additional efforts and strategies to remove them from the reclaimed water. Therefore, the on-line monitoring of NH$_{3}$-N and colour are usually practiced in reclaimed water facilities for assisting the removal of both substances.However, the conventional on-line analyzers are wet-chemistry-based, and the measurement takes time. The limitation creates a potential issue: there may not be sufficient time for the downstream chlorine dosing system to respond to sudden surges in color and ammonia levels. To tackle this challenge, this thesis work developed time-variant models based on machine learning to predict the NH$_{3}$-N concentrations and colour levels in the reclaimed water three hours into the future. For the training dataset, the NH$_{3}$-N and colour data were collected by an on-line analyzer and a customized auto-sampling spectrophotometer, respectively. Both are installed in a reclaimed water treatment facility in Hong Kong. Baseline models for forecasting ammonia concentrations and colour levels were first developed with five machine learning algorithms. Long Short-Term Memory (LSTM) was found to be the most effective algorithm, with the lowest MSE values of 0.0405 and 0.0148 for ammonia and colour forecasting models, respectively. In the training processes, novel data pre-processing methods and feature engineering techniques were implemented to enhance predictive model performance. The data pre-processing methods were proved to enhance the quality of training datasets and improved the performance of ammonia and colour forecasting models by reducing the MSE values by 4.2\% and 8.1\%. The feature engineering results supported that the daily fluctuations in NH$_{3}$-N and colour have correlations with the urban water consumption patterns. This finding further enhanced the NH$_{3}$-N and colour forecasting model performance by reducing MSE by 8.9\% and 28.6\% compared to baseline models. The established models can be used to assist the disinfection control strategies based on the model predictions with the use of traditional process control systems. This research offers novel methods and feature engineering processes for NH$_{3}$-N concentrations and colour levels forecasting in reclaimed water for treatment optimization.

\end{abstract}

