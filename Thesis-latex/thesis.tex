%%%%%%%%%%%%%%%%%%%%%%%%%%%%%%%%%%%%%%%%%%%%%%%%%%%%%%%%%%%%%%%%%%%%%%%%%
%                                                                       %
%              A template file for usage with ustthesis.cls             %
%                                                                       %
%%%%%%%%%%%%%%%%%%%%%%%%%%%%%%%%%%%%%%%%%%%%%%%%%%%%%%%%%%%%%%%%%%%%%%%%%

\documentclass{ustthesis}

\usepackage{amsmath, amssymb}
\usepackage{algorithm}
\usepackage{algorithmic}
\usepackage{color,graphicx}
\usepackage{siunitx}
\usepackage{soul}
\usepackage{graphics} % for pdf, bitmapped graphics files
\usepackage{subcaption}
\newtheorem{proof}{Proof}
\usepackage{bookmark}
\usepackage{hyperref} % for better viewing experience  -- added by alan
\usepackage{textcomp}
\usepackage{multirow}
\usepackage{nth}
\usepackage{indentfirst}
\usepackage{siunitx}
\usepackage{changepage}
\usepackage[a4paper, margin=25mm,textheight=247mm,textwidth=145mm]{geometry} % page requriement from the university ---added by Lei.

% Alan: begin the font trial
% Euler for math | Palatino for rm | Helvetica for ss | Courier for tt
% \renewcommand{\rmdefault}{ppl} % rm

% NOTE Lei: rouphly 32 lines for 12pt font size and 1.5 line spacing
\linespread{1.05}

% \usepackage[scaled]{helvet} % ss
% \usepackage{courier} % tt
% \usepackage{euler} % math
% \usepackage{eulervm} % a better implementation of the euler package (not in gwTeX)
% \normalfont
% \usepackage[T1]{fontenc}
% Alan: end the font trial


\DeclareMathOperator*{\argmax}{\arg\!\max}
\DeclareMathOperator*{\argmin}{\arg\!\min}

\newcommand{\red}[1]{#1}
\newcommand{\tab}[1]{\hspace{3mm}}

% NOTE Lei: personal habit for math symbols.
\newcommand{\bx}{\mathbf{x}}
\newcommand{\bX}{\mathbf{X}}
\newcommand{\by}{\mathbf{y}}
\newcommand{\bY}{\mathbf{Y}}
\newcommand{\bD}{\mathbf{D}}
\newcommand{\bE}{\mathbf{E}}
\newcommand{\ba}{\mathbf{a}}
\newcommand{\bs}{\mathbf{s}}
\newcommand{\bn}{\mathbf{n}}
\newcommand{\bI}{\mathbf{I}}
\newcommand{\bsigma}{\mathbf{\sigma}}
\newcommand{\cS}{\mathcal{S}}
\newcommand{\cX}{\mathcal{X}}
\newcommand{\cA}{\mathcal{A}}
\newcommand{\cB}{\mathcal{B}}
\newcommand{\cD}{\mathcal{D}}
\newcommand{\bu}{\mathbf{u}}
\newcommand{\bv}{\mathbf{v}}
\newcommand{\ts}{\textsuperscript}
\newcommand{\etal}{{\em et al.}}
\newcommand{\norm}[1]{\left\lVert#1\right\rVert}
\DeclareMathOperator{\atantwo}{atan2}
\newcommand{\argminE}{\mathop{\mathrm{argmin}}}

\setcounter{tocdepth}{3}
\setcounter{secnumdepth}{3}

% \renewcommand{\familydefault}{\rmdefault}

% \usepackage{latexsym}
    % Use the "latexsym" package when encountering the following error:
    %   ! LaTeX Error: Command \??? not provided in base LaTeX2e.
% \usepackage{epsf}
    % Use the "epsf" package for including EPS files.

%%%%%%%%%%%%%%%%%%%%%%%%%%%%%%%%%%%%%%%%%%%%%%%%%%%%%%%%%%%%%%%%%%%%%%%%%
%                                                                       %
% Preambles. DO NOT ERASE THEM. Change to suite your particular purpose.%
%                                                                       %
%%%%%%%%%%%%%%%%%%%%%%%%%%%%%%%%%%%%%%%%%%%%%%%%%%%%%%%%%%%%%%%%%%%%%%%%%

\title{Forecasting the Color and Ammonia Concentration in the Reclaimed Water using Deep Learning}  % Title of the thesis.
\author{Ting~Hsi~LEE}     % Author of the thesis.

% NOTE Lei: choose your degree
\degree{\MPhil}             % Degree for which the thesis is.
% %% or
% \degree{\PhD}              % Degree for which the thesis is.

\department{Civil and Environmental Engineering}       % Department to which the thesis

\advisor{Prof.~Chii~SHANG}     % Supervisor.

% NOTE Lei: Uncomment this line if you have a co-supervisor/co-advisor
% \coadvisor{Prof.~Mazi~Zhang}     % Co Supervisor.

\depthead{Prof.~Meimei~Han}    % department head.

\defencedate{2022}{07}{20}      % \defencedate{year}{month}{day}   NOTE Lei: change it to the submission month when you submit your final version.

% NOTE:
%   According to the sample shown in the guidelines, page number is
%   placed below the bottom margin.  However, if the author prefers
%   the page number to be printed above the bottom margin, please
%   activate the following command.
% \PNumberAboveBottomMargin

\graphicspath{{./figure/}}
\begin{document}

%%%%%%%%%%%%%%%%%%%%%%%%%%%%%%%%%%%%%%%%%%%%%%%%%%%%%%%%%%%%%%%%%%%%%%%%%
%                                                                       %
% Now the actual Thesis. The order of output MUST be followed:          %
%                                                                       %
%    1) TITLEPAGE                                                       %
%                                                                       %
% The \maketitle command generates the Title page as well as the        %
% Signature page.                                                       %
%                                                                       %
%%%%%%%%%%%%%%%%%%%%%%%%%%%%%%%%%%%%%%%%%%%%%%%%%%%%%%%%%%%%%%%%%%%%%%%%%
\maketitle

%%%%%%%%%%%%%%%%%%%%%%%%%%%%%%%%%%%%%%%%%%%%%%%%%%%%%%%%%%%%%%%%%%%%%%%%%
%                                                                       %
%     2) DEDICATION (Optional)                                          %
%                                                                       %
% The \dedication and \enddedication commands are optional. If          %
% specified it generates a page for dedication.                         %
%
%%%%%%%%%%%%%%%%%%%%%%%%%%%%%%%%%%%%%%%%%%%%%%%%%%%%%%%%%%%%%%%%%%%%%%%%%

% \dedication
% This is an optional section.
% \enddedication

%%%%%%%%%%%%%%%%%%%%%%%%%%%%%%%%%%%%%%%%%%%%%%%%%%%%%%%%%%%%%%%%%%%%%%%%%
%                                                                       %
%     3) SIGNATURE                                                      %
%                                                                       %
% \signature and \endsignature defines the                              %
% signature page of the Thesis especially for ece.                      %
%                                                                       %
%%%%%%%%%%%%%%%%%%%%%%%%%%%%%%%%%%%%%%%%%%%%%%%%%%%%%%%%%%%%%%%%%%%%%%%%%

\signature
		\begin{figure}[!hb]
		\begin{tabular}{ll}

    %% NOTE Lei:
    %%   In case the space is not enough, try \footnotesize for this part.
		% \footnotesize{Thesis Examination Committee} &  \\[8pt]
		% \footnotesize{1. Prof. xxx (Supervisor)}   &  \footnotesize{Department of Electronic and Computer Engineering} \\[8pt]
		% \footnotesize{2. Prof. xxx}   &  \footnotesize{Department of Electronic and Computer Engineering} \\[8pt]
		% \footnotesize{3. Prof. xxx}   &  \footnotesize{Department of Electronic and Computer Engineering} \\[8pt]
		% \footnotesize{4. Prof. xxx}            &  \footnotesize{Department of Mathematics }\\[8pt]
		% \footnotesize{5. Prof. xxx (External Examiner)}  & \footnotesize{Department of Electrical Engineering and} \\[8pt]
		%                               & \footnotesize{Information Technology} \\[8pt]
		%                               &  \footnotesize{Vienna University of Technology} \\[8pt]
    %% NOTE Lei:
    %%  PhD thesis does not need to record the chairperson here. If you have a co-supervisor, please add the Co-supervisor line manually.
		\end{tabular}
		\end{figure}
\endsignature

%%%%%%%%%%%%%%%%%%%%%%%%%%%%%%%%%%%%%%%%%%%%%%%%%%%%%%%%%%%%%%%%%%%%%%%%%
%                                                                       %
%     3) ACKNOWLEDGMENTS                                                %
%                                                                       %
% \acknowledgments and \endacknowledgments defines the                  %
% Acknowledgments of the author of the Thesis.                          %
%                                                                       %
%%%%%%%%%%%%%%%%%%%%%%%%%%%%%%%%%%%%%%%%%%%%%%%%%%%%%%%%%%%%%%%%%%%%%%%%%

\acknowledgments

I would first express my enormous gratitude to my thesis supervisor Prof. Chii Shang for giving me the opportunity to start my MPhil degress in this research group. I was given chances to expose to different research topics and work with other students, most important, I have learned so much from his mentoring.  He also encourage me to develop the research area I am interested in. I was benefited from his knowledge in this domain and his wisdom in 


I would first express my gratitude to my supervisor Prof. Chii SHANG for giving me patient instruction and continuous support during my MPhil study. He encouraged me to learn from mistakes and move forward in steady steps with new-found wisdom. I benefited a lot from his explicit guidance on critical thinking during research and teaching. He has been an excellent mentor and I sincerely appreciate the opportunity to work with him.

I really appreciate Prof. Guanghao CHEN and Prof. Bo SUN for serving on my thesis examination committee. Special thanks should go to Mr. Ran YIN and Miss. Yingying XIANG for giving critical and constructive suggestions on my research.

Thanks also go to technical staff including Mr. Shing Tak LUI, Mr. Wai Lun Johnson YAU and Mr. Chi Man HO for their assistance in instrumentation operation and maintenance.

With great pleasure, I would like to thank my groupmates, labmates and friends for their assistance and friendship.

Finally, I would like to express my gratitude to my family for their love, support and continuous encouragements.

\endacknowledgments

%%%%%%%%%%%%%%%%%%%%%%%%%%%%%%%%%%%%%%%%%%%%%%%%%%%%%%%%%%%%%%%%%%%%%%%%%
%                                                                       %
%     4) TABLE OF CONTENTS                                              %
%                                                                       %
%%%%%%%%%%%%%%%%%%%%%%%%%%%%%%%%%%%%%%%%%%%%%%%%%%%%%%%%%%%%%%%%%%%%%%%%%

\tableofcontents

%%%%%%%%%%%%%%%%%%%%%%%%%%%%%%%%%%%%%%%%%%%%%%%%%%%%%%%%%%%%%%%%%%%%%%%%%
%                                                                       %
%     5) LIST OF FIGURES (If Any)                                       %
%                                                                       %
%%%%%%%%%%%%%%%%%%%%%%%%%%%%%%%%%%%%%%%%%%%%%%%%%%%%%%%%%%%%%%%%%%%%%%%%%

\listoffigures

%%%%%%%%%%%%%%%%%%%%%%%%%%%%%%%%%%%%%%%%%%%%%%%%%%%%%%%%%%%%%%%%%%%%%%%%%
%                                                                       %
%     6) LIST OF TABLES (If Any)
%                                                                       %
%%%%%%%%%%%%%%%%%%%%%%%%%%%%%%%%%%%%%%%%%%%%%%%%%%%%%%%%%%%%%%%%%%%%%%%%%

\listoftables

%%%%%%%%%%%%%%%%%%%%%%%%%%%%%%%%%%%%%%%%%%%%%%%%%%%%%%%%%%%%%%%%%%%%%%%%%
%                                                                       %
%     7) ABSTRACT                                                       %
%                                                                       %
% \abstract and \endabstract are used to define a short Abstract for    %
% the Thesis.                                                           %
%                                                                       %
%%%%%%%%%%%%%%%%%%%%%%%%%%%%%%%%%%%%%%%%%%%%%%%%%%%%%%%%%%%%%%%%%%%%%%%%%

\begin{abstract}

%Water scarcity is a global challenge. One of the promising ways to mitigate the water resource crisis is via wastewater reclamation. Chlorine is commonly used for reclaimed water disinfection and requires precise dosing to satisfy endorsed quality standards. Ammoniacal nitrogen (NH$_{3}$N) and colour exist in the reclaimed water at concentrations between 0.23 – 5.44 mg N/L and 80 – 150 Hazen units, respectively, and can affect the chlorine demand. Forecasting the reclaimed water quality enables a feedback control system over the disinfection process by predicting the exact chlorine dose required which secures sufficient time to respond to sudden surges in color and ammonia levels. This study developed time-variant models based on machine learning to predict the NH$_{3}$N concentration and colour three hours into the future in the reclaimed water. The NH$_{3}$N data was collected by an online analyzer, and colour data was collected by a customized auto-sampling spectrophotometer, both are installed in the reclaimed water treatment plant in Hong Kong. Long Short-Term Memory (LSTM) was found to be the most effective architecture for training NH$_{3}$N and colour forecasting models. In the training processes, we applied data pre-processing methods and feature engineering, a technique to select or create relevant variables in raw data to enhance predictive model performance. From feature engineering, we discovered that the daily fluctuation in NH$_{3}$N and colour has correlations with the urban water consumption patterns. This finding further enhanced the NH$_{3}$N and colour forecasting model performance by 4.9\% and 5.4\% compared to baseline models. This research work offers novel methods and feature engineering processes for NH$_{3}$N concentration and colour forecasting in reclaimed water for treatment optimization. 

Water scarcity is a global challenge, and one of the promising ways to mitigate the water resource crisis is via wastewater reclamation. Reclaimed water can generate non-potable water to substitute the use of drinking water for irrigation or industrial processes. Water quality and aesthetics are the primary concerns in reclaimed water since undertreated water can pose health risks, and the unpleasant colour is likely to induce public misgiving. Ammoniacal nitrogen (NH$_{3}$-N) and colour substances exist in the reclaimed water and can severely affect the reclaimed water quality in different ways. Chlorine is commonly used for reclaimed water disinfection and requires precise dosing to satisfy endorsed quality standards. However, NH$_{3}$-N consumes chlorine and affects the chlorine dosing. Colour substances do not consume chlorine, but it requires additional efforts and strategies to remove them from the reclaimed water. Therefore, the on-line monitoring of NH$_{3}$-N and colour are usually practised in reclaimed water facilities to assist in the removal of both substances. However, the conventional on-line analyzers are wet-chemistry-based, and the measurement takes time. The limitation creates a potential issue: there may not be sufficient time for the downstream chlorine dosing system to respond to sudden surges in colour and ammonia levels. To tackle this challenge, this thesis work developed time-series models based on machine learning to forecast the NH$_{3}$-N concentrations and colour levels in the reclaimed water three hours into the future. For the training dataset, the NH$_{3}$-N and colour data were collected by an on-line analyzer and a customized auto-sampling spectrophotometer, respectively. Both are installed in a reclaimed water treatment facility in Hong Kong. Baseline models for forecasting ammonia concentrations and colour levels were first developed with five machine learning algorithms. Long Short-Term Memory (LSTM) was found to be the most effective algorithm, with the lowest MSE values of 0.0405 and 0.0148 for ammonia and colour forecasting models, respectively. In the training processes, novel data pre-processing methods and feature engineering techniques were implemented to enhance forecasting model performance. The data pre-processing methods were proved to enhance the quality of training datasets and improve the performance of ammonia and colour forecasting models by reducing the MSE values by 4.2\% and 8.1\%. The feature engineering results supported that the daily fluctuations in NH$_{3}$-N and colour have correlations with the urban water consumption patterns. This finding further enhanced the NH$_{3}$-N and colour forecasting model performance by reducing MSE by 8.9\% and 28.6\% compared to baseline models. The established models can be used to assist the disinfection control strategies based on the model predictions using traditional process control systems. This research offers novel methods and feature engineering processes for NH$_{3}$-N concentrations and colour levels forecasting in reclaimed water for treatment optimization.

\end{abstract}



%%%%%%%%%%%%%%%%%%%%%%%%%%%%%%%%%%%%%%%%%%%%%%%%%%%%%%%%%%%%%%%%%%%%%%%%%
%                                                                       %
%     8) The Actual Contents                                            %
%                                                                       %
% The command \chapters MUST BE USED to ensure that the entire content  %
% of the Thesis is double-spaced (in version 1.0).                      %
%                                                                       %
% However, in version 2.0, \chapters will be automatically added in     %
% the beginning of the first chapter.                                   %
%                                                                       %
%%%%%%%%%%%%%%%%%%%%%%%%%%%%%%%%%%%%%%%%%%%%%%%%%%%%%%%%%%%%%%%%%%%%%%%%%

%%\chapters         % Not necessary with ustthesis.cls (v2.0).

%%%%%%%%%%%%%%%%%%%%%%%%%%%%%%%%%%%%%%%%%%%%%%%%%%%%%%%%%%%%%%%%%%%%%%%%%
%                                                                       %
% Each chapter is defined via the \chapter command. The usual sectional %
% commands of LaTeX are also available.                                 %
%                                                                       %
%%%%%%%%%%%%%%%%%%%%%%%%%%%%%%%%%%%%%%%%%%%%%%%%%%%%%%%%%%%%%%%%%%%%%%%%%


\chapter{Introduction}

\section{Background}

%paragraph 1 Forecasting models play an important role in water quality control in DTPs and WWTPs.
%paragraph 2 Water reclamation-why is it a good choice for solving urban water scarcity
%paragraph 3 Decision-making processing-how does this help water reclamation
%paragraph 4 Deep learning model to replace fuzzy supervisor and machine learning models-the need of using it
AI technologies have been successfully applied to different DWT processes, such as the prediction of the coagulant 
dosage, discrimination of the DBP formation potential, advanced control of membrane fouling, membrane preparation 
and optimization, and water quality prediction. \cite{li_recent_2021}

%%%% Paragraph 1
Forecasting models play an important roles in water quality control in drinking water treatment plants (DTPs) 
and wastewater treatment plants (WWTPs). The need of using forecasting models are becuase the unpredictable 
nature of water quality, and the treatment operations are subjected to the change of water quality to prodcue
effluent complied the government regulation \cite{chen_assessing_2003}.

%%%% Paragraph 2
Forecasting models can also be called time series model becuase the data is consisted of the values and the 
time (need to be further revised). For the well-know time series models are for example, RNN, ... These are 
used to replace the theory-based models, for example Activated Sludege Model (ASM). The difference between 
these two models are, machine learning based models require to learn from historic data, while the thoery-based
models only need to enter the basic operational parameters (e.g., influent flow, tempearture, and pH, etc).

%%%% Paragraph 3
Despite the promising usage and performance of machine learning models, the collection of the data became
the most difficult tasks. Many small scale or old treatment plants do not have the capital or the available
environment for the set-ups of the online sensors to collect data.
Although these are the major issues, it's still possible to train a forecasting model with one input, which 
is also called a self-prediction model. Although the accuracy or stability compared to multi-input models, 
the forecasted results can be used at some cases. To increase the model performance, there are several ways.
Paper included weather data, or perform data-preprocessing methods to improve the model performance.

%%%% Paragraph 4
These solutions (data preprocessing, feature engineering) are not well discussed in this field, also the 
potential of using univariate models are under estimated.

\section{Objectives}
\noindent
The specific objectives of this thesis work are:\\
%should be investigate the effluent water quality in SWHEPP?
(1) To build baseline univariate forecasting models using machine learning and deep learning models.\\
(2) To develop data preprocessing methods for enhancing model forecasting performance.\\
(3) To extract features and hidden relations of water parameters in MBR effluent by analyzing the wastewater collected upstream of the WWTPs.\\
(4) To develop methods for improving performance of forecasting models using the hidden features and relations of the water parameters.

\section{Organization of the thesis}
\chapter{Literature Review}
%\label{sec:ob_rel}

\section{Water reclamation in WWTPs}
\subsection{Reclaimed water for urban water crisis}
%Describe the issue of urban water crisis
%%Statistics from UN or NGOs
%%Comments from some review papers
%%Reports that talk or project HK is also a city with urban water crisis
%Explain the solutions to address urban water crisis
%%The four solutions mentioned by Exall
To address urban water crisis, there are four primary solutions which are rainwater 
harvesting, stormwater runoff, onsite groundwater, and reclaimed water. \cite{exall_integrated_2012}
%%potable and non-potable water uses
%%Talk about the pros of using reclaimation specifically in urban cities, back-up my saying by making HK as an example
%The efforts have been made for generating reclaimed water
%%Use big cities from papers to make good examples to express we need to focus on using reclaimed water

\subsection{Decision-making processes for water quality control}
%Describe the need of using decision-making processes for water quality
%Using examples to show how the decision-making process can benefit water quality control

In this study the new control objectives for the reclaimed water system in 
Shek Wu Hui Effluent Polish Plant have been established: to monitor color 
and ammonia concentration in the MBR effluent and at the same time provide 
a predictive model to assist the disinfection control strategy for disinfecting 
the MBR effluent to meet the endorsed reclaimed water standard.


\subsection{The use of machine learning in model predictive control}
%Describe how mahcine learning become a dominent role or telling people using ml is a trend
%Make exampling of how machine learning really benefit water quality control

\section{Introduction to Deep Learning (DL)}
\subsection{Deep Learning in water quality forecasting}
%why DL is better than ML?

\subsection{Recurrent neural networks (RNNs)}
%why we need to use this?

\section{Environments for developing deep learning models}
\subsection{The use of Python programming languages}
%why python? compared to matlab

\subsection{Build forecasting models with deep learning frameworks}
%what are the frameworks, and why using them



% NOTE Lei:
%    Normally we don't need PART, but I used it in my thesis.
%\input{chapter/s1_obstacle_avoidance}


% NOTE Lei:
%   this is try to set the conlusion as the same level as the PART in bookmark list. Otherwise, in the bookmark list, chapter conlusion will be under the last PART automatically.
\bookmarksetup{startatroot}
\addtocontents{toc}{\bigskip}

%\input{chapter/s5_conclusion}

% \showthe\font
%%%%%%%%%%%%%%%%%%%%%%%%%%%%%%%%%%%%%%%%%%%%%%%%%%%%%%%%%%%%%%%%%%%%%%%%%
%                                                                       %
%      9) BIBLIOGRAPHY                                                  %
%                                                                       %
% This example uses bibtex to generate the required Bibliography. Refer %
% to the % the file ustthesis_test.bib for the entries of the           %
% Bibliography. Note that only the cited entries are printed.           %
%                                                                       %
% If BibTeX is not used to typeset the bibliography, replace the        %
% following line with the \begin{thebibliography} and \end{bibliography}%
% commands (the "thebibliography" environment) to process the           %
% Bibliography.                                                         %
%                                                                       %
%%%%%%%%%%%%%%%%%%%%%%%%%%%%%%%%%%%%%%%%%%%%%%%%%%%%%%%%%%%%%%%%%%%%%%%%%

%%%%%%%%%%%%%%%%%%%%%%%%%%%%%%%%%%%%%%%%%%%%%%%%%%%%%%%%%%%%%%%%%%%%%%%%%
%                                                                       %
% The recommended bibliography style is the IEEE bibliography style.    %
% "ustbib" defines the IEEE bibliography standard with the added        %
% ability of sorting the items by name of author.                       %
%                                                                       %
% If you are not using BibTeX to process your Bibliography, comment out %
% the following line.                                                   %
%                                                                       %
%%%%%%%%%%%%%%%%%%%%%%%%%%%%%%%%%%%%%%%%%%%%%%%%%%%%%%%%%%%%%%%%%%%%%%%%%

\bibliographystyle{plain}

\bibliography{ref}

%%%%%%%%%%%%%%%%%%%%%%%%%%%%%%%%%%%%%%%%%%%%%%%%%%%%%%%%%%%%%%%%%%%%%%%%%
%                                                                       %
%     10) APPENDIX (If Any)                                             %
%                                                                       %
%                                                                       %
%%%%%%%%%%%%%%%%%%%%%%%%%%%%%%%%%%%%%%%%%%%%%%%%%%%%%%%%%%%%%%%%%%%%%%%%%
%\input{chapter/s6_appendices}


\end{document}
